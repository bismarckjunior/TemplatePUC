\documentclass{../CursoLaTeX}

\subtitle{Tabelas (Básico)}

\usepackage{array}
\usepackage{multirow}

\def\hc{\centering\arraybackslash}
\newcolumntype{M}[1]{>{\hc} m{#1}}
\newcolumntype{P}[1]{>{\hc} p{#1}}
\newcolumntype{B}[1]{>{\hc} b{#1}}
\newcolumntype{N}{@{}m{0pt}@{}}

\newcommand{\vcell}[3][c]{\begin{tabular}[#1]{@{}#2@{}}#3\end{tabular}}

\begin{document}

%%%%%%%%%%%%%%%%%%%%%%%%%%%%%%%%%%%%%%%%%%%%%%%%%%%%%%%%%
\begin{frame}{Sumário} 
  \tableofcontents[] 
\end{frame}

%%%%%%%%%%%%%%%%%%%%%%%%%%%%%%%%%%%%%%%%%%%%%%%%%%%%%%%%%
\section{Tabela Simples}

%%%%%%%%%%%%%%%%%%%%%%%%%%%
\begin{frame}[fragile]{Tabela Simples}
O comando \textbf{tabular} possibilita a inclusão de tabelas no documento.

\begin{lstlisting}[style=latex, linewidth=12cm, xleftmargin=2cm]
\documentclass{article}

\begin{document}

  % Inserindo tabela
  \begin{tabular}{ccc}
    A & B & C \\
    1 & 2 & 3 \\
  \end{tabular}

\end{document}
\end{lstlisting}

\end{frame}


%%%%%%%%%%%%%%%%%%%%%%%%%%%
\begin{frame}[fragile,t]{Tabela Simples}
\framesubtitle{Opções de colunas}

Toda tabela no \LaTeX\ precisa ser iniciada da seguinte forma:

\begin{lstlisting}[style=latex, linewidth=6cm]
\begin{tabular}{ COL }
\end{lstlisting}

\begin{columns}
\begin{column}{8cm}
Opções do COL:

\begin{tabular}{cl}
\hline
 \bf COL  & \bf Descrição         \\
\hline
      r    & Alinhado à direita    \\
      c    & Alinhado ao centro    \\
      l    & Alinhado à esquerda   \\
      |    & Borda lateral simples \\
      ||   & Borda lateral dupla   \\
 *\{N\}\{COL\} & Repete COL, N vezes   \\
\hline
\end{tabular}
\end{column}

\begin{column}{6.2cm}
Exemplo:
\begin{lstlisting}[style=latex]
\begin{tabular}{|l c r|}
  A & B & C \\
  1 & 2 & 3 \\
\end{tabular}

\begin{tabular}{*{3}{c}}
  A & B & C \\
  1 & 2 & 3 \\
\end{tabular}
\end{lstlisting}
\end{column}
\end{columns}
\end{frame}

%%%%%%%%%%%%%%%%%%%%%%%%%%%
\begin{frame}[fragile,t]{Tabela Simples}
\framesubtitle{Exemplo}

\makebox[0.65\linewidth][l]{Código \LaTeX:}%
\makebox[0.35\linewidth][l]{Saída:}% 
\vfill

\begin{LTXexample}[style=example,width=0.35\linewidth]
\begin{tabular}{ r *{2}{c} l }
  A A & B B & C C & D D \\
  1   & 4   & 7   & 10 \\
  2   & 5   & 8   & 11 \\ 
  3   & 6   & 9   & 12 \\
\end{tabular}
\end{LTXexample}

\end{frame}

%%%%%%%%%%%%%%%%%%%%%%%%%%%%%%%%%%%%%%%%%%%%%%%%%%%%%%%%%
\section{Bordas}

%%%%%%%%%%%%%%%%%%%%%%%%%%%
\begin{frame}[fragile,t]{Bordas}
\framesubtitle{Linhas verticais (| e ||)}

\makebox[0.65\linewidth][l]{Código \LaTeX:}%
\makebox[0.35\linewidth][l]{Saída:}% 
\vfill

\begin{LTXexample}[style=example,width=0.35\linewidth]
\begin{tabular}{||c|c|c||}
  A A & B B & C C \\
  1   & 4   & 7   \\
  2   & 5   & 8   \\ 
  3   & 6   & 9   \\
\end{tabular}
\end{LTXexample}

\end{frame}

%%%%%%%%%%%%%%%%%%%%%%%%%%%
\begin{frame}[fragile,t]{Bordas}
\framesubtitle{Linhas verticais (| e ||)}

\makebox[0.65\linewidth][l]{Código \LaTeX:}%
\makebox[0.35\linewidth][l]{Saída:}% 
\vfill

\begin{LTXexample}[style=example,width=0.35\linewidth]
\begin{tabular}{||c|c|c||}
  A A & B B & C C \\
  1   & 4   & 7   \\
  2   & 5   & 8   \\ 
  3   & 6   & 9   \\
\end{tabular}
\end{LTXexample}

\end{frame}

%%%%%%%%%%%%%%%%%%%%%%%%%%%
\begin{frame}[fragile,t]{Bordas}
\framesubtitle{Linhas horizontais (\bs hline)}

\makebox[0.65\linewidth][l]{Código \LaTeX:}%
\makebox[0.35\linewidth][l]{Saída:}% 
\vfill

\begin{LTXexample}[style=example,width=0.35\linewidth]
\begin{tabular}{ccc}
\hline
  A A & B B & C C \\
\hline
  1   & 4   & 7   \\
  2   & 5   & 8   \\ 
  3   & 6   & 9   \\
\hline
\end{tabular}
\end{LTXexample}

\end{frame}

%%%%%%%%%%%%%%%%%%%%%%%%%%%
\begin{frame}[fragile,t]{Bordas}
\framesubtitle{Linhas horizontais (\bs cline{i-j})}

\makebox[0.65\linewidth][l]{Código \LaTeX:}%
\makebox[0.35\linewidth][l]{Saída:}% 
\vfill

\begin{LTXexample}[style=example,width=0.35\linewidth]
\begin{tabular}{c c c}
\cline{1-2}
  A A & B B & C C \\
\cline{2-3}
  1   & 4   & 7   \\
  2   & 5   & 8   \\ 
  3   & 6   & 9   \\
\hline
\end{tabular}
\end{LTXexample}

\end{frame}

%%%%%%%%%%%%%%%%%%%%%%%%%%%
\begin{frame}[fragile,t]{Bordas}
\framesubtitle{Espessura das linhas}

\makebox[0.65\linewidth][l]{Código \LaTeX:}%
\makebox[0.35\linewidth][l]{Saída:}% 
\vfill

\begin{LTXexample}[style=example,width=0.35\linewidth]
\setlength{\arrayrulewidth}{1mm}
\begin{tabular}{|c|c|c|}
\hline
  A A & B B & C C \\
\hline
  1   & 4   & 7   \\
  2   & 5   & 8   \\ 
  3   & 6   & 9   \\
\hline
\end{tabular}
\end{LTXexample}

\end{frame}


%%%%%%%%%%%%%%%%%%%%%%%%%%%%%%%%%%%%%%%%%%%%%%%%%%%%%%%%%
\section{Alinhamento Horizontal}

%%%%%%%%%%%%%%%%%%%%%%%%%%%
\begin{frame}[fragile,t]{Alinhamento Horizontal}
\framesubtitle{Opções l, c, r}

\makebox[0.65\linewidth][l]{Código \LaTeX:}%
\makebox[0.35\linewidth][l]{Saída:}% 
\vfill

\begin{LTXexample}[style=example,width=0.35\linewidth]
\begin{tabular}{|l|c|r|}
\hline
  A A & B B & C C \\
\hline
  1   & 4   & 7   \\
  2   & 5   & 8   \\ 
  3   & 6   & 9   \\
\hline
\end{tabular}
\end{LTXexample}

\end{frame}

%%%%%%%%%%%%%%%%%%%%%%%%%%%
\begin{frame}[fragile,t]{Comprimento Fixo de Colunas}
\framesubtitle{Pacote \textit{array}}

Pacote necessário para fixar o comprimento das colunas: \textit{array}

\begin{lstlisting}[style=latex, linewidth=12cm]
\documentclass{article}

% Pacotes
\usepackage{array}     % tabela

\begin{document}
  % Inserindo tabela com colunas fixas
  \begin{tabular}{ |m{1cm}|m{1cm}|}
    A  &  B \\
    1  &  2 \\
  \end{tabular}

\end{document}
\end{lstlisting}

\end{frame}

%%%%%%%%%%%%%%%%%%%%%%%%%%%
\begin{frame}[fragile,t]{Comprimento Fixo de Colunas}
\framesubtitle{Opção m\{\ldots\}}

\makebox[0.65\linewidth][l]{Código \LaTeX:}%
\makebox[0.35\linewidth][l]{Saída:}% 
\vfill

\begin{LTXexample}[style=example,width=0.35\linewidth]
\begin{tabular}{|m{1cm}| c | m{15mm}|}
\hline
  A A & B B & C C \\
\hline
  1   & 4   & 7   \\
  2   & 5   & 8   \\
  3   & 6   & 9   \\
\hline
\end{tabular}
\end{LTXexample}

\end{frame}

%%%%%%%%%%%%%%%%%%%%%%%%%%%
\begin{frame}[fragile,t]{Espaço Entre Colunas}
\framesubtitle{Comando \textit{tabcolsep}}

\makebox[0.65\linewidth][l]{Código \LaTeX:}%
\makebox[0.35\linewidth][l]{Saída:}% 
\vfill

\begin{LTXexample}[style=example,width=0.35\linewidth]
\setlength{\tabcolsep}{0mm}
\begin{tabular}{|c|c|c|}
\hline
  A A & B B & C C \\
\hline
  1   & 4   & 7   \\
  2   & 5   & 8   \\
  3   & 6   & 9   \\
\hline
\end{tabular}
\end{LTXexample}

\end{frame}

%%%%%%%%%%%%%%%%%%%%%%%%%%%
\begin{frame}[fragile,t]{Espaço Entre Colunas}
\framesubtitle{Comando \textit{tabcolsep}}

\makebox[0.65\linewidth][l]{Código \LaTeX:}%
\makebox[0.35\linewidth][l]{Saída:}% 
\vfill

\begin{LTXexample}[style=example,width=0.35\linewidth]
\setlength{\tabcolsep}{2mm}
\begin{tabular}{|c|c|c|}
\hline
  A A & B B & C C \\
\hline
  1   & 4   & 7   \\
  2   & 5   & 8   \\
  3   & 6   & 9   \\
\hline
\end{tabular}
\end{LTXexample}

\end{frame}


%%%%%%%%%%%%%%%%%%%%%%%%%%%%%%%%%%%%%%%%%%%%%%%%%%%%%%%%%
\section{Alinhamento Vertical}

%%%%%%%%%%%%%%%%%%%%%%%%%%%
\begin{frame}[fragile,t]{Alinhamento Vertical}
\framesubtitle{Pacote \textit{array}}

Pacote necessário para alinhar verticalmente: \textit{array}

\begin{lstlisting}[style=latex, linewidth=12cm]
\documentclass{article}

% Pacotes
\usepackage{array}     % tabela

\begin{document}
  % Inserindo tabela com colunas fixas
  \begin{tabular}{ |m{1cm}|p{1cm}|}
    A  &  B \\
    1  &  2 \\
  \end{tabular}

\end{document}
\end{lstlisting}

\end{frame}

%%%%%%%%%%%%%%%%%%%%%%%%%%%
\begin{frame}[fragile,t]{Alinhamento Vertical}
\framesubtitle{Opções de colunas}

É possível expandir as opções padrões de colunas utilizando o pacote \textit{array}:

\begin{lstlisting}[style=latex, linewidth=6cm]
\usepackage{array}
\end{lstlisting}

\vspace{0.5cm}

\begin{columns}
\begin{column}{8cm}
Opções do COL:

\begin{tabular}{cl}
\hline
 \bf COL  & \bf Descrição         \\
\hline
    m\{1cm\}  & Alinhado verticalmente ao centro \\
    p\{1cm\}  & Alinhado verticalmente ao topo\\
    b\{1cm\}  & Alinhado verticalmente à baixo  \\
\hline
\end{tabular}
\end{column}

\begin{column}{6.2cm}
Exemplo:
\begin{lstlisting}[style=latex]
\begin{tabular}{|m{2cm}cc|}
  A & B & C \\
  1 & 2 & 3 \\
\end{tabular}
\end{lstlisting}
\end{column}
\end{columns}
\end{frame}

%%%%%%%%%%%%%%%%%%%%%%%%%%%
\begin{frame}[fragile,t]{Alinhamento Vertical}
\framesubtitle{Opção de colunas: m}

\makebox[0.65\linewidth][l]{Código \LaTeX:}%
\makebox[0.35\linewidth][l]{Saída:}% 
\vfill

\begin{LTXexample}[style=example,width=0.35\linewidth]
\begin{tabular}{|m{4mm}| c | m{2cm}|}
\hline
  A A & B B & C C \\
\hline
  1   & 4   & 7   \\
  2   & 5   & 8   \\
  3   & 6   & 9   \\
\hline
\end{tabular}
\end{LTXexample}

\end{frame}

%%%%%%%%%%%%%%%%%%%%%%%%%%%
\begin{frame}[fragile,t]{Alinhamento Vertical}
\framesubtitle{Opção de colunas: p}

\makebox[0.65\linewidth][l]{Código \LaTeX:}%
\makebox[0.35\linewidth][l]{Saída:}% 
\vfill

\begin{LTXexample}[style=example,width=0.35\linewidth]
\begin{tabular}{|p{4mm}| c | p{2cm}|}
\hline
  A A & B B & C C \\
\hline
  1   & 4   & 7   \\
  2   & 5   & 8   \\
  3   & 6   & 9   \\
\hline
\end{tabular}
\end{LTXexample}

\end{frame}

%%%%%%%%%%%%%%%%%%%%%%%%%%%
\begin{frame}[fragile,t]{Alinhamento Vertical}
\framesubtitle{Opção de colunas: b}

\makebox[0.65\linewidth][l]{Código \LaTeX:}%
\makebox[0.35\linewidth][l]{Saída:}% 
\vfill

\begin{LTXexample}[style=example,width=0.35\linewidth]
\begin{tabular}{|b{4mm}| c | b{2cm}|}
\hline
  A A & B B & C C \\
\hline
  1   & 4   & 7   \\
  2   & 5   & 8   \\
  3   & 6   & 9   \\
\hline
\end{tabular}
\end{LTXexample}

\end{frame}


%%%%%%%%%%%%%%%%%%%%%%%%%%%
\begin{frame}[fragile,t]{Alinhamento Vertical}
\framesubtitle{Mais opções de colunas}

É possível expandir as opções padrões de colunas ainda mais adicionando os comandos:

\begin{lstlisting}[style=latex, linewidth=12cm]
\def\hc{\centering\arraybackslash}
\newcolumntype{M}[1]{>{\hc} m{#1}}
\newcolumntype{P}[1]{>{\hc} p{#1}}
\newcolumntype{B}[1]{>{\hc} b{#1}}
\end{lstlisting}

\vspace{3mm}

\begin{columns}
\begin{column}{8cm}
Opções do COL:

\begin{tabular}{cl}
\hline
 \bf COL  & \bf Descrição         \\
\hline
    M\{1cm\}  & Alinhado ao centro \\
    P\{1cm\}  & Alinhado ao centro e ao topo\\
    B\{1cm\}  & Alinhado ao centro e à baixo  \\
\hline
\end{tabular}
\end{column}

\begin{column}{6.2cm}
Exemplo:
\begin{lstlisting}[style=latex]
\begin{tabular}{|M{2cm}cc|}
  A & B & C \\
  1 & 2 & 3 \\
\end{tabular}
\end{lstlisting}
\end{column}
\end{columns}
\end{frame}

%%%%%%%%%%%%%%%%%%%%%%%%%%%
\begin{frame}[fragile,t]{Alinhamento Vertical}
\framesubtitle{Mais opções de colunas: M}

\makebox[0.65\linewidth][l]{Código \LaTeX:}%
\makebox[0.35\linewidth][l]{Saída:}% 
\vfill

\begin{LTXexample}[style=example,width=0.35\linewidth]
\begin{tabular}{|M{4mm}|c|M{16mm}|}
\hline
  A A & B B & C C \\
\hline
  1   & 4   & 7   \\
  2   & 5   & 8   \\ 
  3   & 6   & 9   \\
\hline
\end{tabular}
\end{LTXexample}

\end{frame}

%%%%%%%%%%%%%%%%%%%%%%%%%%%
\begin{frame}[fragile,t]{Alinhamento Vertical}
\framesubtitle{Mais opções de colunas: P}

\makebox[0.65\linewidth][l]{Código \LaTeX:}%
\makebox[0.35\linewidth][l]{Saída:}% 
\vfill

\begin{LTXexample}[style=example,width=0.35\linewidth]
\begin{tabular}{|P{4mm}|c| P{16mm}|}
\hline
  A A & B B & C C \\
\hline
  1   & 4   & 7   \\
  2   & 5   & 8   \\ 
  3   & 6   & 9   \\
\hline
\end{tabular}
\end{LTXexample}

\end{frame}


%%%%%%%%%%%%%%%%%%%%%%%%%%%
\begin{frame}[fragile,t]{Alinhamento Vertical}
\framesubtitle{Mais opções de colunas: B}

\makebox[0.65\linewidth][l]{Código \LaTeX:}%
\makebox[0.35\linewidth][l]{Saída:}% 
\vfill

\begin{LTXexample}[style=example,width=0.35\linewidth]
\begin{tabular}{|B{4mm}|c|B{16mm}|}
\hline
  A A & B B & C C \\
\hline
  1   & 4   & 7   \\
  2   & 5   & 8   \\ 
  3   & 6   & 9   \\
\hline
\end{tabular}
\end{LTXexample}

\end{frame}


%%%%%%%%%%%%%%%%%%%%%%%%%%%
\begin{frame}[fragile,t]{Espaçamento Vertical}
\framesubtitle{Comando \textit{arraystretch}}
\vspace{-3mm}
\makebox[0.65\linewidth][l]{Código \LaTeX:}%
\makebox[0.35\linewidth][l]{Saída:}% 
\vfill

\begin{LTXexample}[style=example,width=0.35\linewidth]
\renewcommand{\arraystretch}{1.0}
\begin{tabular}{|l|l|l|}
\hline
  A A & B B & C C \\
\hline
  1   & 4   & 7   \\
\hline
  2   & 5   & 8   \\ 
\hline
  3   & 6   & 9   \\
\hline
\end{tabular}
\end{LTXexample}

\end{frame}


%%%%%%%%%%%%%%%%%%%%%%%%%%%
\begin{frame}[fragile,t]{Espaçamento Vertical}
\framesubtitle{Comando \textit{arraystretch}}
\vspace{-3mm}
\makebox[0.65\linewidth][l]{Código \LaTeX:}%
\makebox[0.35\linewidth][l]{Saída:}% 
\vfill

\begin{LTXexample}[style=example,width=0.35\linewidth]
\renewcommand{\arraystretch}{2.0}
\begin{tabular}{*{3}{|l}|}
\hline
  A A & B B & C C \\
\hline
  1   & 4   & 7   \\
\hline
  2   & 5   & 8   \\ 
\hline
  3   & 6   & 9   \\
\hline
\end{tabular}
\end{LTXexample}

\end{frame}

%%%%%%%%%%%%%%%%%%%%%%%%%%%
\begin{frame}[fragile,t]{Espaçamento Vertical}
\framesubtitle{Comando \bs \bs [\ldots]}

É possível expandir o espaço vertical de uma coluna adicionando o valor desejado entre colchetes no final da linha ou adicionando linhas vazias:

\vspace{5mm}
Exemplo:
\begin{lstlisting}[style=latex, linewidth=12cm]
\begin{tabular}{ccc}
  A A & B B & C C \\ [4mm] % Adicionando 4mm na linha
  1   & 4   & 7   \\
      &     &     \\       % Adicionando linha vazia
  2   & 5   & 8   \\
      &     &     \\       % Adicionando linha vazia
  3   & 6   & 9   \\
\end{tabular}
\end{lstlisting}

\end{frame}

%%%%%%%%%%%%%%%%%%%%%%%%%%%
\begin{frame}[fragile,t]{Espaçamento Vertical}
\framesubtitle{Comando \bs \bs [\ldots]}
\vspace{-3mm}
\makebox[0.65\linewidth][l]{Código \LaTeX:}%
\makebox[0.35\linewidth][l]{Saída:}% 
\vfill

\begin{LTXexample}[style=example,width=0.35\linewidth]
\begin{tabular}{*{3}{|l}|}
\hline
  A A & B B & C C \\ [4mm] 
\hline
  1   & 4   & 7   \\
  2   & 5   & 8   \\
  3   & 6   & 9   \\
\hline
\end{tabular}
\end{LTXexample}

\end{frame}

%%%%%%%%%%%%%%%%%%%%%%%%%%%
\begin{frame}[fragile,t]{Espaçamento Vertical}
\framesubtitle{Linhas vazias}
\vspace{-3mm}
\makebox[0.65\linewidth][l]{Código \LaTeX:}%
\makebox[0.35\linewidth][l]{Saída:}% 
\vfill

\begin{LTXexample}[style=example,width=0.35\linewidth]
\begin{tabular}{*{3}{|l}|}
\hline
      &     &     \\ 
  A A & B B & C C \\
      &     &     \\ 
\hline
  1   & 4   & 7   \\
  2   & 5   & 8   \\
  3   & 6   & 9   \\
\hline
\end{tabular}
\end{LTXexample}

\end{frame}

%%%%%%%%%%%%%%%%%%%%%%%%%%%
\begin{frame}[fragile,t]{Espaçamento Vertical}
\framesubtitle{Linhas vazias dimensionadas}
\vspace{-3mm}
\makebox[0.65\linewidth][l]{Código \LaTeX:}%
\makebox[0.35\linewidth][l]{Saída:}% 
\vfill

\begin{LTXexample}[style=example,width=0.35\linewidth]
\begin{tabular}{*{3}{|l}|}
\hline
      &     &     \\ [-2mm]
  A A & B B & C C \\
      &     &     \\ [-2mm]
\hline
  1   & 4   & 7   \\
  2   & 5   & 8   \\
  3   & 6   & 9   \\
\hline
\end{tabular}
\end{LTXexample}

\end{frame}

%%%%%%%%%%%%%%%%%%%%%%%%%%%
\begin{frame}[fragile,t]{Espaçamento Vertical}
\framesubtitle{Erro no comando \bs \bs [\ldots]}

\makebox[0.65\linewidth][l]{Código \LaTeX:}%
\makebox[0.35\linewidth][l]{Saída:}% 
\vfill

\begin{LTXexample}[style=example,width=0.35\linewidth]
\begin{tabular}{|M{4mm}|c|M{16mm}|}
\hline
  A A & B B & C C \\ [1cm]
\hline
  1   & 4   & 7   \\
  2   & 5   & 8   \\
  3   & 6   & 9   \\
\hline
\end{tabular}
\end{LTXexample}

\end{frame}

%%%%%%%%%%%%%%%%%%%%%%%%%%%
\begin{frame}[fragile,t]{Espaçamento Vertical}
\framesubtitle{Coluna nula}

O commando \bs\bs [\ldots] pode nem sempre funcionar da maneira esperada.
Para corrigir esse problema é possível adicionar uma coluna nula
criada pelo comando:

\begin{lstlisting}[style=latex, linewidth=8cm]
\newcolumntype{N}{@{}m{0pt}@{}}
\end{lstlisting}

\vspace{5mm}
Exemplo:
\begin{lstlisting}[style=latex, linewidth=12cm]
\begin{tabular}{|M{4mm}|c|M{16mm}|N}
  A A & B B & C C &\\ [1cm]
  1   & 4   & 7   &\\
  2   & 5   & 8   &\\ 
  3   & 6   & 9   &\\
\end{tabular}
\end{lstlisting}

\end{frame}

%%%%%%%%%%%%%%%%%%%%%%%%%%%
\begin{frame}[fragile,t]{Espaçamento Vertical}
\framesubtitle{Coluna nula}

\makebox[0.65\linewidth][l]{Código \LaTeX:}%
\makebox[0.35\linewidth][l]{Saída:}% 
\vfill

\begin{LTXexample}[style=example,width=0.35\linewidth]
\begin{tabular}{|M{4mm}|c|M{16mm}|N}
\hline
  A A & B B & C C &\\ [1cm]
\hline
  1   & 4   & 7   &\\
  2   & 5   & 8   &\\ 
  3   & 6   & 9   &\\
\hline
\end{tabular}
\end{LTXexample}

\end{frame}

%%%%%%%%%%%%%%%%%%%%%%%%%%%
\begin{frame}[fragile,t]{Espaçamento Vertical}
\framesubtitle{Linhas vazias dimensionadas}
\vspace{-3mm}
\makebox[0.65\linewidth][l]{Código \LaTeX:}%
\makebox[0.35\linewidth][l]{Saída:}% 
\vfill

\begin{LTXexample}[style=example,width=0.35\linewidth]
\begin{tabular}{|M{4mm}|c|M{16mm}|}
\hline
  A A & B B & C C \\ 
      &     &     \\ [5mm]
\hline
  1   & 4   & 7   \\
  2   & 5   & 8   \\
  3   & 6   & 9   \\
\hline
\end{tabular}
\end{LTXexample}

\end{frame}

%%%%%%%%%%%%%%%%%%%%%%%%%%%
\begin{frame}[fragile,t]{Quebra de linha}
\framesubtitle{Comando \textit{vcell}}

Para que seja possível uma quebra de linha dentro de uma célula,
é necessário definir o comando \textit{vcell}:

\begin{lstlisting}[style=latex, linewidth=8cm]
\newcommand{\vcell}[3][c]{
  \begin{tabular}[#1]{@{}#2@{}}#3
  \end{tabular}
}
\end{lstlisting}


\begin{columns}

\begin{column}{6.5cm}
Uso:
\begin{lstlisting}[style=latex, linewidth=6cm]
\vcell[POS]{COL}{TXT}
\end{lstlisting}

\textbf{POS} (opcional): alinhamento vertical

\textbf{COL}: alinhamento horizontal

\textbf{TXT}: texto da célula

\end{column}
\begin{column}{4cm}
Alinhamento vertical:
\vspace{2mm}

\begin{tabular}{cl}
\hline
POS   & Alinhamento \\
\hline
\bf c & Centro \\
 t    & Topo   \\
 b    & Baixo  \\
\hline
\end{tabular}

\end{column}
\begin{column}{4cm}
Alinhamento horizontal:
\vspace{2mm}

\begin{tabular}{cl}
\hline
COL   & Alinhamento \\
\hline
 c    & Centro   \\
 r    & Direita  \\
 l    & Esquerda \\
\hline
\end{tabular}
\end{column}
\end{columns}

\end{frame}

%%%%%%%%%%%%%%%%%%%%%%%%%%%
\begin{frame}[fragile,t]{Quebra de linha}
\framesubtitle{Comando \textit{vcell}}
\vspace{-3mm}
\makebox[0.65\linewidth][l]{Código \LaTeX:}%
\makebox[0.35\linewidth][l]{Saída:}% 
\vfill

\begin{LTXexample}[style=example,width=0.35\linewidth]
\begin{tabular}{|M{4mm}|c|M{16mm}|}
\hline
  A & B B & \vcell{l}{CCC \\ C} \\ 
\hline
  1   & 4   & 7   \\
  2   & 5   & 8   \\
  3   & 6   & 9   \\
\hline
\end{tabular}
\end{LTXexample}

\end{frame}

%%%%%%%%%%%%%%%%%%%%%%%%%%%
\begin{frame}[fragile,t]{Quebra de linha}
\framesubtitle{Comando \textit{vcell}}
\vspace{-3mm}
\makebox[0.65\linewidth][l]{Código \LaTeX:}%
\makebox[0.35\linewidth][l]{Saída:}% 
\vfill

\begin{LTXexample}[style=example,width=0.35\linewidth]
\begin{tabular}{|c|c|c|}
\hline
  A & B B & \vcell{c}{CCC \\ C} \\ 
\hline
  1   & 4   & 7   \\
  2   & 5   & 8   \\
  3   & 6   & 9   \\
\hline
\end{tabular}
\end{LTXexample}

\end{frame}

%%%%%%%%%%%%%%%%%%%%%%%%%%%
\begin{frame}[fragile,t]{Quebra de linha}
\framesubtitle{Comando \textit{vcell}}
\vspace{-3mm}
\makebox[0.65\linewidth][l]{Código \LaTeX:}%
\makebox[0.35\linewidth][l]{Saída:}% 
\vfill

\begin{LTXexample}[style=example,width=0.35\linewidth]
\begin{tabular}{|c|c|c|}
\hline
  A & B B & \vcell[t]{c}{CCC \\ C} \\ 
\hline
  1   & 4   & 7   \\
  2   & 5   & 8   \\
  3   & 6   & 9   \\
\hline
\end{tabular}
\end{LTXexample}

\end{frame}

%%%%%%%%%%%%%%%%%%%%%%%%%%%
\begin{frame}[fragile,t]{Quebra de linha}
\framesubtitle{Comando \textit{vcell}}
\vspace{-3mm}
\makebox[0.65\linewidth][l]{Código \LaTeX:}%
\makebox[0.35\linewidth][l]{Saída:}% 
\vfill

\begin{LTXexample}[style=example,width=0.35\linewidth]
\begin{tabular}{|c|c|c|}
\hline
  A & B B & \vcell[b]{c}{CCC \\ C} \\ 
\hline
  1   & 4   & 7   \\
  2   & 5   & 8   \\
  3   & 6   & 9   \\
\hline
\end{tabular}
\end{LTXexample}

\end{frame}


%%%%%%%%%%%%%%%%%%%%%%%%%%%%%%%%%%%%%%%%%%%%%%%%%%%%%%%%%
\section{Mesclagem}

%%%%%%%%%%%%%%%%%%%%%%%%%%%
\begin{frame}[fragile,t]{Mesclagem Horizontal}
\framesubtitle{Comando \textit{multicolumn}}

Para mesclar células horizontalmente deve-se utilizar o comando 
\textit{multicolumn}.

% \begin{lstlisting}[style=latex, linewidth=8cm]
% \newcommand{\vcell}[3][c]{
%   \begin{tabular}[#1]{@{}#2@{}}#3
%   \end{tabular}
% }
% \end{lstlisting}

\vspace{7mm}

\begin{columns}[t]

\begin{column}{6.5cm}
Uso:
\begin{lstlisting}[style=latex, linewidth=6cm]
\multicolumn{NUM}{COL}{TXT}
\end{lstlisting}

\textbf{NUM}: número de colunas

\textbf{COL}: opção de coluna (ex: c, |c|, \ldots)

\textbf{TXT}: texto da célula

\end{column}
\begin{column}{5.5cm}
Exemplo:
\begin{lstlisting}[style=latex, linewidth=8cm]
\begin{tabular}{|c|c|c|}
\hline
  \multicolumn{2}{c}{AABB} & C C \\ 
\hline
  1   & 4   & 7   \\
  2   & 5   & 8   \\
  3   & 6   & 9   \\
\hline
\end{tabular}
\end{lstlisting}

\end{column}
\end{columns}

\end{frame}

%%%%%%%%%%%%%%%%%%%%%%%%%%%
\begin{frame}[fragile,t]{Mesclagem Horizontal}
\framesubtitle{Comando \textit{multicolumn}}
\vspace{-3mm}
\makebox[0.65\linewidth][l]{Código \LaTeX:}%
\makebox[0.35\linewidth][l]{Saída:}% 
\vfill

\begin{LTXexample}[style=example,width=0.35\linewidth]
\begin{tabular}{|c|c|c|}
\hline
  \multicolumn{2}{c}{AABB} & C C \\ 
\hline
  1   & 4   & 7   \\
  2   & 5   & 8   \\
  3   & 6   & 9   \\
\hline
\end{tabular}
\end{LTXexample}

\end{frame}

%%%%%%%%%%%%%%%%%%%%%%%%%%%
\begin{frame}[fragile,t]{Mesclagem Horizontal}
\framesubtitle{Comando \textit{multicolumn}}
\vspace{-3mm}
\makebox[0.65\linewidth][l]{Código \LaTeX:}%
\makebox[0.35\linewidth][l]{Saída:}% 
\vfill

\begin{LTXexample}[style=example,width=0.35\linewidth]
\begin{tabular}{|c|c|c|}
\hline
  \multicolumn{2}{|l|}{AABB} & C C \\ 
\hline
  1   & 4   & 7   \\
  2   & 5   & 8   \\
  3   & 6   & 9   \\
\hline
\end{tabular}
\end{LTXexample}

\end{frame}

%%%%%%%%%%%%%%%%%%%%%%%%%%%
\begin{frame}[fragile,t]{Mesclagem Horizontal}
\framesubtitle{Comando \textit{multicolumn}}
\vspace{-3mm}
\makebox[0.65\linewidth][l]{Código \LaTeX:}%
\makebox[0.35\linewidth][l]{Saída:}% 
\vfill

\begin{LTXexample}[style=example,width=0.35\linewidth]
\begin{tabular}{|M{1cm}|c|c|}
\hline
  \multicolumn{1}{|l|}{A} & B & C C \\ 
\hline
  1   & 4   & 7   \\
  2   & 5   & 8   \\
  3   & 6   & 9   \\
\hline
\end{tabular}
\end{LTXexample}

* O comando \textit{multcolumn} também pode ser utilizado para alinhar horizontamente uma célula.

\end{frame}

%%%%%%%%%%%%%%%%%%%%%%%%%%%
\begin{frame}[fragile,t]{Mesclagem Vertical}
\framesubtitle{Comando \textit{multirow}}

Para mesclar células horizontalmente deve-se utilizar o comando 
\textit{multicolumn} importando o pacote \textit{multirow}.

\begin{lstlisting}[style=latex, linewidth=6cm]
\usepackage{multirow}
\end{lstlisting}

\vspace{2mm}

\begin{columns}[t]

\begin{column}{6.5cm}
Uso:
\begin{lstlisting}[style=latex, linewidth=6.5cm]
\multirow{NUM}{LAR}[ESP]{TXT}
\end{lstlisting}

\textbf{NUM}: número de linhas

\textbf{TAM}: largura da célula

\textbf{ESP} (opcional): espaço vertical

\textbf{TXT}: texto da célula

\end{column}
\begin{column}{4.5cm}
Exemplo:
\begin{lstlisting}[style=latex, linewidth=8cm]
\begin{tabular}{|c|c|c|}
  A A & B B & C C \\ 
 \multirow{3}{1cm}{123} & 4 & 7 \\
     & 5   & 8   \\
     & 6   & 9   \\
\end{tabular}
\end{lstlisting}

\end{column}
\end{columns}

\end{frame}

%%%%%%%%%%%%%%%%%%%%%%%%%%%
\begin{frame}[fragile,t]{Mesclagem Vertical}
\framesubtitle{Comando \textit{multirow}}
\vspace{-3mm}
\makebox[0.65\linewidth][l]{Código \LaTeX:}%
\makebox[0.35\linewidth][l]{Saída:}% 
\vfill

\begin{LTXexample}[style=example,width=0.35\linewidth]
\begin{tabular}{|c|c|c|}
\hline
  A A & B B & C C \\ 
\hline
  \multirow{3}{*}{123} & 4 & 7 \\
     & 5   & 8   \\
     & 6   & 9   \\
\hline
\end{tabular}
\end{LTXexample}

\end{frame}

%%%%%%%%%%%%%%%%%%%%%%%%%%%
\begin{frame}[fragile,t]{Mesclagem Vertical}
\framesubtitle{Comando \textit{multirow}}
\vspace{-3mm}
\makebox[0.65\linewidth][l]{Código \LaTeX:}%
\makebox[0.35\linewidth][l]{Saída:}% 
\vfill

\begin{LTXexample}[style=example,width=0.35\linewidth]
\begin{tabular}{|c|c|c|}
\hline
  A A & B B & C C \\ 
\hline
  \multirow{3}{2cm}{123} & 4 & 7 \\
     & 5   & 8   \\
     & 6   & 9   \\
\hline
\end{tabular}
\end{LTXexample}

\end{frame}

%%%%%%%%%%%%%%%%%%%%%%%%%%%
\begin{frame}[fragile,t]{Mesclagem Vertical}
\framesubtitle{Comando \textit{multirow}}
\vspace{-3mm}
\makebox[0.65\linewidth][l]{Código \LaTeX:}%
\makebox[0.35\linewidth][l]{Saída:}% 
\vfill

\begin{LTXexample}[style=example,width=0.35\linewidth]
\begin{tabular}{|c|c|c|}
\hline
  A A & B B & C C \\ 
\hline
  \multirow{3}{2cm}[4mm]{123} & 4 & 7 \\
     & 5   & 8   \\
     & 6   & 9   \\
\hline
\end{tabular}
\end{LTXexample}

\end{frame}

%%%%%%%%%%%%%%%%%%%%%%%%%%%
\begin{frame}[fragile,t]{Mesclagem Vertical}
\framesubtitle{Comando \textit{multirow}}
\vspace{-3mm}
\makebox[0.65\linewidth][l]{Código \LaTeX:}%
\makebox[0.35\linewidth][l]{Saída:}% 
\vfill

\begin{LTXexample}[style=example,width=0.35\linewidth]
\begin{tabular}{|c|c|c|}
\hline
  A A & B B & C C \\ 
\hline
  \multirow{3}{2cm}[-4mm]{123} & 4 & 7 \\
     & 5   & 8   \\
     & 6   & 9   \\
\hline
\end{tabular}
\end{LTXexample}

\end{frame}

%%%%%%%%%%%%%%%%%%%%%%%%%%%
\begin{frame}[fragile,t]{Mesclagem Mista}
% \framesubtitle{Comando \textit{multirow}}
\vspace{-3mm}
\makebox[0.65\linewidth][l]{Código \LaTeX:}%
\makebox[0.35\linewidth][l]{Saída:}% 
\vfill

\begin{LTXexample}[style=example,width=0.35\linewidth]
\begin{tabular}{ c c c }
\hline
  A A & B B & C C \\ 
\hline
 \multicolumn{2}{c}{
    \multirow{3}{*}{123} } & 7 \\
  & & 8   \\
  & & 9   \\
\hline
\end{tabular}
\end{LTXexample}

\end{frame}

%%%%%%%%%%%%%%%%%%%%%%%%%%%
\begin{frame}[fragile,t]{Mesclagem Mista}
% \framesubtitle{Comando \textit{multirow}}
\vspace{-3mm}
\makebox[0.65\linewidth][l]{Código \LaTeX:}%
\makebox[0.35\linewidth][l]{Saída:}% 
\vfill

\begin{LTXexample}[style=example,width=0.35\linewidth]
\begin{tabular}{|c|c|c|}
\hline
  A A & B B & C C \\ 
\hline
 \multicolumn{2}{|c|}{
    \multirow{3}{*}{123} } & 7 \\
  & & 8   \\
  & & 9   \\
\hline
\end{tabular}
\end{LTXexample}

\end{frame}

%%%%%%%%%%%%%%%%%%%%%%%%%%%
\begin{frame}[fragile,t]{Mesclagem Mista}
% \framesubtitle{Comando \textit{multirow}}
\vspace{-3mm}
\makebox[0.65\linewidth][l]{Código \LaTeX:}%
\makebox[0.35\linewidth][l]{Saída:}% 
\vfill

\begin{LTXexample}[style=example,width=0.35\linewidth]
\begin{tabular}{|c|c|c|}
\hline
  A A & B B & C C \\ 
\hline
 \multicolumn{2}{|c|}{
    \multirow{3}{*}{123} } & 7 \\
  \multicolumn{2}{|c|}{} & 8   \\
  \multicolumn{2}{|c|}{} & 9   \\
\hline
\end{tabular}
\end{LTXexample}

\end{frame}

%%%%%%%%%%%%%%%%%%%%%%%%%%%
\begin{frame}[fragile,t]{Resumo}
% \framesubtitle{Opções de coluna}
% \vspace{-3mm}

\begin{columns}

\begin{column}{5cm}
Uso:
\begin{lstlisting}[style=latex, linewidth=5cm]
\begin{tabular}{ COL }
 TABLE
\end{tabular}
\end{lstlisting}

\end{column}

\begin{column}{8cm}

\begin{tabular}{cl}
\hline
 \bf COL  & \bf Descrição         \\
\hline
      r    & Alinhado à direita    \\
      c    & Alinhado ao centro    \\
      l    & Alinhado à esquerda   \\
      |    & Borda lateral simples \\
      ||   & Borda lateral dupla   \\
 *\{N\}\{COL\} & Repete COL, N vezes   \\
    m\{1cm\}  & Alinhado verticalmente ao centro \\
    p\{1cm\}  & Alinhado verticalmente ao topo\\
    b\{1cm\}  & Alinhado verticalmente à baixo  \\
    M\{1cm\}  & Alinhado ao centro \\
    P\{1cm\}  & Alinhado ao centro e ao topo\\
    B\{1cm\}  & Alinhado ao centro e à baixo  \\
    N         & Coluna neutra \\
\hline
\end{tabular}
\end{column}
\end{columns}
\end{frame}

%%%%%%%%%%%%%%%%%%%%%%%%%%%
\begin{frame}[fragile]{Resumo}

\begin{tabular}{ll}
\hline
 \bf TABLE     & \bf Descrição         \\
\hline
     \&           & Separador de coluna    \\
  \bs\bs          & Separador de linha     \\
 \bs\bs[1cm]      & Aumentar altura da linha   \\
  \bs hline       & Linha horizontal \\
\bs cline\{i-j\}  & Linha horizontal parcial \\
\bs vcell[POS]\{COL\}\{TXT\} & Célula com quebra de linha   \\
\bs multirow\{NUM\}\{LAR\}[ESP]\{TXT\}  & Mesclagem vertical \\
\bs multicolumn\{NUM\}\{COL\}\{TXT\} & Mesclagem horizontal   \\
\hline
\end{tabular}

\end{frame}

\end{document}
