% 1. Estilo do documento
\documentclass[mestrado]{TemplatePUC}

% 2. Referencias biliograficas
\usepackage{csquotes} % evita mensagem de erro no babel
\usepackage[style=abnt, justify]{biblatex}
\addbibresource{bibliografia.bib}

% 3. Pacotes
\usepackage{multirow}

% 4. Definições globais
\imprimirlista{fig}{tab}{nom}

% 5. Dados do documento
\titulo{Título do Trabalho em Portugês}
\titulous{ Title in English }

\autor{Bismarck Gomes Souza Júnior}
\autorR{Souza Júnior, Bismarck Gomes}

\orientador{Beltrano Matos Figueira}{PhD.}
\orientadorR{Figueira, Beltrano Matos}

\dia{30}
\mes{abril}
\ano{2017}

\departamento{Engenharia Civil e Ambiental}
\programa{Engenharia Civil}
\centro{Centro Técnico Científico}

% 6. Banca examinadora
\banca{
  \membrodabanca{Fulano de Tal}{PhD.}
    {Departamento de Engenharia Civil e Ambiental}{PUC-Rio}

  \membrodabanca{Ciclano Almeida Pinto}{Dr.}
    {Departamento de Engenharia Civil}{UFRJ}

  \coordenador{José Eugenio Leal}{Prof.}
}

% 7. Currículo
\curriculo{ Graduou-se em \ldots }

% 8. Ficha catalográfica.
\colorido{true} 
\CDD{624}

% 9. Palavras-chave
\prechaves{
  \prechave{Engenharia civil}
}
\chaves{
  \chave{geomecânica de reservatórios}
  \chave{acoplamento termo-hidro-mecânico}
  \chave{acoplamento de malhas}
}
\chavesus{
  \chave{reservoir geomechanics}
  \chave{thermo-hydro-mechanical coupling}
  \chave{mesh coupling}
}

% 10. Resumo e abstract
\resumo{ Na simulação numérica de \ldots }
\resumous{ Numerical simulation is \ldots }

% 11. Dedicatória e agradecimentos
\agradecimentos{ A Deus por tudo \ldots }

% 12. Epigrafe
\epigrafe{ O sucesso é ir de fracasso em fracasso sem perder entusiasmo.}
\epigrafeautor{Winston Churchill}

% 13. Documento
\begin{document}

  %!TEX root = ./Thesis.tex

%%
%% Bismarck Gomes <bismarck@puc-rio.br>
%%

\chapter{Introdução} 
\label{chap:Introducao}

%%%%%%%%%%%%%%%%%%%%%%%%%%%%%%%%%%%%%%%%%%%%%%%%%%%%%%%%%%%%%%%%%%%%%%%%%%%%%%%%
\section{Escopo do Problema} 
\label{sec:Escopo-do-Problema}

\lipsum[7-8]


%%%%%%%%%%%%%%%%%%%%%%%%%%%%%%%%%%%%%%%%%%%%%%%%%%%%%%%%%%%%%%%%%%%%%%%%%%%%%%%%
\section{Objetivos}
\label{sec:Objetivos}

\lipsum[10-12]


%%%%%%%%%%%%%%%%%%%%%%%%%%%%%%%%%%%%%%%%%%%%%%%%%%%%%%%%%%%%%%%%%%%%%%%%%%%%%%%%
\section{Organização do Documento}
\label{sec:Organização-do-Documento}

\lipsum[13-15]


  %!TEX root = ./PreThesis.tex

%%
%% Bismarck Gomes <bismarck@puc-rio.br>
%%

\chapter{Revisão Bibliográfica}
\label{chap:Revisão-Bibliográfica}

\lipsum[15]

%%%%%%%%%%%%%%%%%%%%%%%%%%%%%%%%%%%%%%%%%%%%%%%%%%%%%%%%%%%%%%%%%%%%%%%%%%%%%%%%
\section{Artigos Tecnológicos}
\label{sec:Artigos-Tecnológicos}

\lipsum[16-20]


%%%%%%%%%%%%%%%%%%%%%%%%%%%%%%%%%%%%%%%%%%%%%%%%%%%%%%%%%%%%%%%%%%%%%%%%%%%%%%%%
\section{Artigos Científicos}
\label{sec:Artigos-Científicos}

Donec molestie, magna ut luctus ultrices, tellus arcu nonummy velit, sit
amet pulvinar elit justo et mauris. In pede. Maecenas euismod elit eu erat.
Aliquam augue wisi, facilisis congue, suscipit in, adipiscing et, ante. In justo.
Cras lobortis neque ac ipsum. Nunc fermentum massa at ante. Donec orci
tortor, egestas sit amet, ultrices eget, venenatis eget, mi. Maecenas vehicula
leo semper est. Mauris vel metus. Aliquam erat volutpat. In rhoncus sapien ac
tellus. Pellentesque ligula \cite{biot1957}.

Cras dapibus, augue quis scelerisque ultricies, felis dolor placerat sem, id
porta velit odio eu elit. Aenean interdum nibh sed wisi. Praesent sollicitudin
vulputate dui. Praesent iaculis viverra augue. Quisque in libero. Aenean
gravida lorem vitae sem ullamcorper cursus. Nunc adipiscing rutrum ante.
Nunc ipsum massa, faucibus sit amet, viverra vel, elementum semper, orci.
Cras eros sem, vulputate et, tincidunt id, ultrices eget, magna. Nulla varius
ornare odio. Donec accumsan mauris sit amet augue. Sed ligula lacus, laoreet
non, aliquam sit amet, iaculis tempor, lorem. Suspendisse eros. Nam porta,
leo sed congue tempor, felis est ultrices eros, id mattis velit felis non metus.
Curabitur vitae elit non mauris varius pretium. Aenean lacus sem, tincidunt
ut, consequat quis, porta vitae, turpis. Nullam laoreet fermentum urna. Proin
iaculis lectus \cite{terzaghi1943}.

Sed mattis, erat sit amet gravida malesuada, elit augue egestas diam,
tempus scelerisque nunc nisl vitae libero. Sed consequat feugiat massa. Nunc
porta, eros in eleifend varius, erat leo rutrum dui, non convallis lectus orci
ut nibh. Sed lorem massa, nonummy quis, egestas id, condimentum at, nisl.
Maecenas at nibh. Aliquam et augue at nunc pellentesque ullamcorper. Duis
nisl nibh, laoreet suscipit, convallis ut, rutrum id, enim. Phasellus odio. Nulla
nulla elit, molestie non, scelerisque at, vestibulum eu, nulla. Ut odio nisl,
facilisis id, mollis et, scelerisque nec, enim. Aenean sem leo, pellentesque sit
amet, scelerisque sit amet, vehicula pellentesque, sapien \cite{anderson1989}.

\lipsum[40]


%%%%%%%%%%%%%%%%%%%%%%%%%%%%%%%%%%%%%%%%%%%%%%%%%%%%%%%%%%%%%%%%%%%%%%%%%%%%%%%%
\section{Crítica aos Trabalho Existentes}
\label{sec:Crítica-aos-Trabalho-Existentes}

\lipsum[25-29]

  
  \printbibliography

  \appendix
  %!TEX root = ./Masters.tex

%%
%% Bismarck Gomes <bismarck@puc-rio.br>
%%

\chapter{Arquivos de Entrada}
\label{chap:Arquivos-de-Entrada}

\lipsum[30]


%%%%%%%%%%%%%%%%%%%%%%%%%%%%%%%%%%%%%%%%%%%%%%%%%%%%%%%%%%%%%%%%%%%%%%%%%%%%%%%%
\section{Simulação A}
\label{sec:Simulação-A}

\lipsum[31-33]


%%%%%%%%%%%%%%%%%%%%%%%%%%%%%%%%%%%%%%%%%%%%%%%%%%%%%%%%%%%%%%%%%%%%%%%%%%%%%%%%
\section{Simulação B}
\label{sec:Simulação-B}

\lipsum[35-40]


\end{document}
