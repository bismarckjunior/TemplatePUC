%!TEX root = ./PreThesis.tex

%%
%% Bismarck Gomes <bismarck@puc-rio.br>
%%

\chapter{Metodologia}
\label{chap:Metodologia}

Conforme discutido no Capítulo \ref{chap:Revisão-Bibliográfica} \ldots
\lipsum[1]
%%%%%%%%%%%%%%%%%%%%%%%%%%%%%%%%%%%%%%%%%%%%%%%%%%%%%%%%%%%%%%%%%%%%%%%%%%%%%%%%
\section{Figuras}
\label{sec:Figuras}

A Figura \ref{fig:puc.png} ilustra um exemplo de como referenciar uma figura, 
enquanto que a Figura \ref{fig:puc2}, como utilizar subfiguras.

\begin{figure}[ht]
  \centering
  \includegraphics[scale=1.0]{figs/puc.png}
  \caption{Logo da PUC-Rio}
  \label{fig:puc.png}
 \end{figure}

\begin{figure}[ht]
  \centering
  \subfigure[Legenda 1]{\includegraphics[scale=0.3]{figs/puc.png}}
  \hspace{1cm}
  \subfigure[Legenda 2]{\includegraphics[scale=0.4]{figs/puc.png}}
  \caption{Logo da PUC-Rio em forma de subfigura}
  \label{fig:puc2}
 \end{figure}

%%%%%%%%%%%%%%%%%%%%%%%%%%%%%%%%%%%%%%%%%%%%%%%%%%%%%%%%%%%%%%%%%%%%%%%%%%%%%%%%
\section{Equações}
\label{sec:Equações}
 
Exemplos de equações:

\begin{itemize}
	\item{Equação simples: eq. (\ref{eq:simples})}
	\item{Equação grande: eq. (\ref{eq:grande})}
	\item{Equação alinhada: eq. (\ref{eq:alinhada})}
\end{itemize}

Seguem as equações, começando com uma relação bem simples entre os lados de um triângulo retângulo escaleno:
\begin{equation} \label{eq:simples}
  a^2 = b^2 + c^2,
 \end{equation}
onde $a$ é a hipotenusa; $b$, o menor cateto; e $c$, o maior cateto.
\nomenclature[A]{$a$}{Hipotenusa}
\nomenclature[A]{$b$}{Menor cateto}
\nomenclature[A]{$c$}{Maior cateto}

Já uma equação maior, que ocupa mais de uma linha, é necessário um tratamento especial. Por exemplo, a série de Taylor de uma função $f$ em torno do ponto $x=x_0$ dada por:
\begin{multline} \label{eq:grande}
  f(x) = f(x_0)+f'(x_0)\frac{\left(x-x_0\right)}{1!} +
  			 f''(x_0)\frac{\left(x-x_0\right)^2}{2!} + \\
  			 f'''(x_0)\frac{\left(x-x_0\right)^3}{3!} 
  			 +\ldots+f^{(n)}(x_0)\frac{\left(x-x_0\right)^n}{n!}.
	\nomenclature[A]{$f$}{Função qualquer}
 \end{multline}

Por outro lado, o desenvolvimento de uma equação pode ser escrita na forma:
\begin{align} \label{eq:alinhada}
	A &= \pi r^2     \nonumber\\
	  &= \pi (10)^2  \nonumber\\
	  &= 100\pi 
  \end{align}

Da mesma forma, é possível escrever um sistema de equações:
\begin{eqnarray} \label{eq:tmp74}
    x+y+z &=& 1 \\
    3x-2y-4z &=& 10 \\
    3x-2y &=& 20
 \end{eqnarray}
ou ainda:

 \begin{equation} \label{eq:tmp81}
  \begin{cases}
    x+y +z= 1 \\
    3x-2y -4z= 10 \\
    3x-2y = 20
  \end{cases}
 \end{equation}


%%%%%%%%%%%%%%%%%%%%%%%%%%%%%%%%%%%%%%%%%%%%%%%%%%%%%%%%%%%%%%%%%%%%%%%%%%%%%%%%
\section{Tabelas}
\label{sec:Tabelas}

Seguem o exemplo de duas tabelas, Tabela \ref{tab:Primeira-tabela} e Tabela \ref{tab:Segunda-tabela}.

\begin{table}[ht]
	\centering
	\caption{Primeira tabela}
	\label{tab:Primeira-tabela}
	% \vspace{0.5em}
	\begin{tabular}{ clc } 
	\hline
	Células 	& Lugar    & Valor \\ 
	\hline
	c1   			& Caxias   & 90,0 \\ 
	c2 				& Mesquita & 59,0 \\ 
	\hline
	\end{tabular}
\end{table}

\begin{table}[ht]
	\centering
	\caption{Segunda tabela}
	\label{tab:Segunda-tabela}
	% \vspace{0.5em}
	\begin{tabular}{ clc } 
	\hline
	Células 	& Lugar    & Valor \\ 
	\hline
	\multirow{2}{1cm}{\centering RJ} 
	    			& Caxias   & 90,0 \\ 
	   				& Mesquita & 59,0 \\ 
	\hline
	\end{tabular}
\end{table}

\begin{table}[ht]
	\centering
	\caption{Segunda tabela}
	% \vspace{0.5em}
	\begin{tabular}{ clc } 
	\hline
	Células 	& Lugar    & Valor \\ 
	\hline
	\multirow{2}{1cm}{\centering RJ} 
	    			& Caxias   & 90,0 \\ 
	   				& Mesquita & 59,0 \\ 
	\hline
	\end{tabular}
\end{table}

\begin{table}[ht]
	\centering
	\caption{Segunda tabela}
	% \vspace{0.5em}
	\begin{tabular}{ clc } 
	\hline
	Células 	& Lugar    & Valor \\ 
	\hline
	\multirow{2}{1cm}{\centering RJ} 
	    			& Caxias   & 90,0 \\ 
	   				& Mesquita & 59,0 \\ 
	\hline
	\end{tabular}
\end{table}

\begin{table}[ht]
	\centering
	\caption{Segunda tabela}
	% \vspace{0.5em}
	\begin{tabular}{ clc } 
	\hline
	Células 	& Lugar    & Valor \\ 
	\hline
	\multirow{2}{1cm}{\centering RJ} 
	    			& Caxias   & 90,0 \\ 
	   				& Mesquita & 59,0 \\ 
	\hline
	\end{tabular}
\end{table}


%%%%%%%%%%%%%%%%%%%%%%%%%%%%%%%%%%%%%%%%%%%%%%%%%%%%%%%%%%%%%%%%%%%%%%%%%%%%%%%%
\section{Referências}
\label{chap:Referências}

Segundo \textcite{anderson1989}, a terra é levemente arredondada. E de acordo com \textapud{biot1957}{righetto2016} é possível chegar ao Japão, cavando um buraco bem profundo no Brazil \cite{aavatsmark2007}.

Porém, lembre-se de sempre vericar a fonte\footcite{edwards1998} de suas informações para não cometer o equívoco\footnote{Texto descritivo da seleção} de divulgar uma informação errada \apud{terzaghi1943}{zienkiewicz1999}.
\iffalse
\fi